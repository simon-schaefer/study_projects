\section{Discussion}
\label{sec:discussion}

Future improvements include making the genetic agent more robust by looking one move ahead and factoring that in to decide the best current move. Also, one could try to find a state space representation suitable for playing Tetris with the Q-Learning approach. Overall, we are pretty satisfied with our results, especially considering the amount of time and effort we invested in exploring the Q-Learning and the Autoencoder approach.  We were able to scale our algorithm to use big data, since the genetic algorithm has an inherent structure for parallelization. Our implementation using multithreading was able to execute 86 generations of the genetic agent with a population size of 100 in about 5 hours using the 10 cores on the NSCC cluster achieving a mean cleared rows of about 118,587. 
\newline
\newline
The genetic algorithm and PSO optimization parameters required extensive tuning, and given the limited time, we focused on tuning the heuristics for crossing over and mutation rate, along with different population sizes, since this helps in achieving a better population in fewer generations. We also tried tuning the weights of velocity and coefficients of influence of the local and global best results for the population using PSO. But this only led to marginal improvement in results. The genetic algorithm evolves quite steadily, but slowly in the solution space, while the PSO occasionally gives good results. Hence, we thought of combining the two so that any good set of weights in the solution space obtained randomly could be converged upon. This is evident of the fact that genetic algorithm tends to localize to a suboptimal solution and exploit the solution space, while PSO tends to explore it. A higher velocity coefficient explores the search space more. So, the search problem is essentially balancing between exploitation and exploration, which could have been improved by applying PSO on a fraction of the population. We observe a large variability in the results, as is inherent in mutation, selection threshold of parents and initial set of random weights of the population and velocities imparted to the features. 


